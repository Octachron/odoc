\section{Module \ocamlinlinecode{Alias}}\label{package-test+u+package+++ml-module-Alias}%
\label{package-test+u+package+++ml-module-Alias-module-Foo+u++u+X}\ocamlcodefragment{\begin{ocamlkeyword}module\end{ocamlkeyword}
 \hyperref[package-test+u+package+++ml-module-Alias-module-Foo+u++u+X]{\ocamlinlinecode{Foo\_\allowbreak{}\_\allowbreak{}X}}}\ocamlcodefragment{ : \begin{ocamlkeyword}sig\end{ocamlkeyword}
}\begin{ocamlindent}\label{package-test+u+package+++ml-module-Alias-module-Foo+u++u+X-type-t}\ocamlcodefragment{\begin{ocamlkeyword}type\end{ocamlkeyword}
 t = \hyperref[xref-unresolved]{\ocamlinlinecode{int}}}\begin{ocamlindent}Module Foo\_\_X documentation. This should appear in the documentation for the alias to this module 'X'\end{ocamlindent}%
\medbreak
\end{ocamlindent}%
\ocamlcodefragment{\begin{ocamlkeyword}end\end{ocamlkeyword}
}\\
\label{package-test+u+package+++ml-module-Alias-module-X}\ocamlcodefragment{\begin{ocamlkeyword}module\end{ocamlkeyword}
 \hyperref[package-test+u+package+++ml-module-Alias-module-X]{\ocamlinlinecode{X}}}\ocamlcodefragment{ : \begin{ocamlkeyword}sig\end{ocamlkeyword}
 .\allowbreak{}.\allowbreak{}.\allowbreak{} \begin{ocamlkeyword}end\end{ocamlkeyword}
}\\

\section{Module \ocamlinlinecode{Alias.\allowbreak{}X}}\label{package-test+u+package+++ml-module-Alias-module-X}%
\label{package-test+u+package+++ml-module-Alias-module-X-type-t}\ocamlcodefragment{\begin{ocamlkeyword}type\end{ocamlkeyword}
 t = int}\begin{ocamlindent}Module Foo\_\_X documentation. This should appear in the documentation for the alias to this module 'X'\end{ocamlindent}%
\medbreak



